%%%%%%%%%%%%%%%%%%%%%%%%%%%%%%%%%%%%%%%%%
% Journal Article LaTeX Template Version 1.3 (9/9/13)
%
% This template has been downloaded from:
% http://www.LaTeXTemplates.com
%
% Original author: Frits Wenneker (http://www.howtotex.com)
%
% License: CC BY-NC-SA 3.0
% (http://creativecommons.org/licenses/by-nc-sa/3.0/)
%
%%%%%%%%%%%%%%%%%%%%%%%%%%%%%%%%%%%%%%%%%

% ----------------------------------------------------------------------------------------
% PACKAGES AND OTHER DOCUMENT CONFIGURATIONS
% ----------------------------------------------------------------------------------------

\documentclass[twoside]{article}


\usepackage[sc]{mathpazo} % Use the Palatino font
\usepackage[T1]{fontenc} % Use 8-bit encoding that has 256 glyphs
\linespread{1.05} % Line spacing - Palatino needs more space between lines
\usepackage{microtype} % Slightly tweak font spacing for aesthetics

\usepackage[hmarginratio=1:1,top=32mm,columnsep=20pt]{geometry} % Document margins
\usepackage{multicol} % Used for the two-column layout of the document
\usepackage[hang,
small,labelfont=bf,up,textfont=it,up]{caption} % Custom captions under/above floats in tables or figures
\usepackage{booktabs} % Horizontal rules in tables
\usepackage{float} % Required for tables and figures in the multi-column environment - they need to be placed in specific locations with the [H] (e.g. \begin{table}[H])

\usepackage{graphicx}

\usepackage{hyperref} % For hyperlinks in the PDF

\usepackage{lettrine} % The lettrine is the first enlarged letter at the beginning of the text
\usepackage{paralist} % Used for the compactitem environment which makes bullet points with less space between them

\usepackage{abstract} % Allows abstract customization
\renewcommand{\abstractnamefont}{\normalfont\bfseries} % Set the "Abstract" text to bold
\renewcommand{\abstracttextfont}{\normalfont\small\itshape} % Set the abstract itself to small italic text

\usepackage{titlesec} % Allows customization of titles
\renewcommand\thesection{\Roman{section}} % Roman numerals for the sections
\renewcommand\thesubsection{\Roman{subsection}} % Roman numerals for subsections
\titleformat{\section}[block]{\large\scshape\centering}{\thesection.}{1em}{} % Change the look of the section titles
\titleformat{\subsection}[block]{\large}{\thesubsection.}{1em}{} % Change the look of the section titles

\usepackage{fancyhdr} % Headers and footers
\pagestyle{fancy} % All pages have headers and footers
\fancyhead{} % Blank out the default header
\fancyfoot{} % Blank out the default footer
\fancyhead[C]{EOSC 555 Assignment 1 $\bullet$ November 2014} % Custom header text
\fancyfoot[RO,LE]{\thepage} % Custom footer text

\usepackage{amsmath}
% ----------------------------------------------------------------------------------------
% TITLE SECTION
% ----------------------------------------------------------------------------------------

\title{\vspace{-15mm}\fontsize{24pt}{10pt}\selectfont\textbf{Discretization of the Helmholtz equation}} % Article title

\author{
  \large
  \textsc{Ben Bougher}\\[2mm] % Your name
  \normalsize Department of Earth, Ocean, and Atmospheric Science\\
  \normalsize University of British Columbia \\ % Your institution
  \normalsize \href{mailto:bbougher@eos.ubc.ca}{bbougher@eos.ubc.ca} % Your email address
  \vspace{-5mm}
}
\date{}

% ----------------------------------------------------------------------------------------

\begin{document}

\maketitle % Insert title

\thispagestyle{fancy} % All pages have headers and footers

% ----------------------------------------------------------------------------------------
% ABSTRACT
% ----------------------------------------------------------------------------------------

\begin{abstract}

  Many physical systems are modelled mathematically by PDE's that
  require numerical solutions. An accurate computational model hinges
  on a discrete representation of both the data model and the
  differential operators.  This manuscript describes the
  discretization of the Helmholtz equation in 1,2 and 3 dimensions,
  and provides documentation for the corresponding JULIA
  implementation.

\end{abstract}

% ----------------------------------------------------------------------------------------
% ARTICLE CONTENTS
% ----------------------------------------------------------------------------------------

\begin{multicols}{2} % Two-column layout throughout the main article text

  \section{Introduction}

  The Helmholtz equation is ubiquitous across many fields of science
  as it models the spatial component of the wave equation. In
  particular interest to the author, the Helmholtz equation is used to
  model the physics of seismic p-waves
  \begin{equation}
    \label{eq:helmholtz}
    \nabla \cdot \rho \nabla U + \omega ^2 mU = -q
  \end{equation}
  where $\rho$ is the density of the earth, $U$ is the scalar
  p-wavefield, $\omega = 2\pi f$, $q$ is the source term, and
  \begin{equation}
    \label{eq:mc}
    m=1/c^2
  \end{equation}
  where $c$ is the p-wave velocity profile of the earth.

  Seismic imaging is the process of reconstructing the tomography of
  the subsurface from reflected waves measured at the surface. If the
  velocity profile of the earth is known, seismic images are created
  by migrating wavelet arrivals in the measured wavefield to the
  geospatial location of the reflector. Unfortunately, velocity
  profiles for seismic acquisitions are sparsely sampled and require
  time-consuming human estimates in order to form a useful
  image. Forward modelling the wavefield $U$ from the estimated
  velocity model $m$ and comparing to the measured field can provide
  insight into the accuracy of the estimate. Minimizing this data
  misfit with respect to the velocity model is the basis of
  Full-Waveform Inversion (FWI), a computationally-intensive method of
  providing high-resolution velocity models of the subsurface.

  This manuscript first reviews the weak-form discretization of the
  Helmholtz equation, then describes the Julia implementation in 1, 2,
  and 3-D. The convergence of the discretization is first demonstrated
  using the method of fictitious sources, then the discrete Helmholtz
  solver is tested qualitatively on the Marmousi model. The paper
  concludes with plans for future work.


  % ------------------------------------------------

\section{Discretization}
The strong form of the Helmholtz equation, given in equation
\ref{eq:helmholtz}, imposes smoothness constraints on the derivatives
of $U$, which can be relaxed using the weak-form representation:
\begin{equation}
  \label{eq:helmholtz_soft_U}
  \int\limits_V f\nabla \cdot \rho edv+ \int\limits_V f\omega ^2 mUdv= -\int\limits_V fqdv
\end{equation}
\begin{equation}
  \label{eq:helmholtz_soft_e}
  \int\limits_V g\nabla Udv = -\int\limits_V gedv
\end{equation}
where
\begin{equation}
  \label{eq:helmholtz_e}
  e= \nabla U 
\end{equation}
and $f$ and $g$ are arbitrary smooth test functions.  Integrating by
parts and applying a zero Neumann boundary condition yields:
\begin{equation}
  \label{eq:U_parts}
  -\int\limits_V \rho e \cdot \nabla fdv +  \int\limits_V f\omega ^2 mUdv= -\int\limits_V fqdv
\end{equation}
\begin{equation}
  \label{eq:e_parts}
  \int\limits_V U\nabla gdv=\int\limits_V gedv
\end{equation}
The integrals in \ref{eq:U_parts} and \ref{eq:e_parts} can be
discretized by summing the estimated integral of each element of a
mesh of size $N$. The integral of each element is approximated by
calculating the value at the cell centre and multiplying by the cell
volume.

Sampling $U$, $q$, $m$, and $f$ at the nodes of the cells, $e$ and $g$
at the cell edges, and the model parameters $\rho$ and $m$ at cell
centres yields:
\begin{align}
  -s^T V \rho \bigodot Av^e &(e\bigodot Gf) + \omega ^2 s^T Avf \bigodot U \nonumber \\
  \label{eq:discrete_1}
  &= -VAv(f\bigodot q)
\end{align}
\begin{equation}
  \label{eq:discrete_2}
  s^T VAv^v (U\bigodot Gf) = s^T VAv^v (g\bigodot q)
\end{equation}
where
\begin{compactitem}
\item $s^T$ is a summation vector
\item $V$ is a volume scaling operator
\item $Av^e$ is an averaging operator which averages a vector field
  from edges to cell centres and sums the components
\item $Av$ is an averaging operator that averages cell nodes to cell
  centres
\item $G$ is a gradient operator that performs a scaled difference of
  adjacent nodes
\item $Av^v$ averages vector fields from edges to cell centres
\item $U$, $e$, $\rho$, $m$, $f$, and $g$ are flattened
  representations of the data matrices
\item $\bigodot$ is the Hadarmard product
\end{compactitem}
Equations \ref{eq:discrete_1} and \ref{eq:discrete_2} can be
simplified to one expression independent of the test functions $f$ and
$g$:
\begin{align}
  \label{eq:discrete_helmholtz}
  -G^Tdiag(Av^eV\rho)&GU + diag(Av^Tw^2m)U \nonumber \\ =
  &-diag(Av^Tv^T)q
\end{align}

% ------------------------------------------------

\section{Software}
This section documents the JULIA implementation of the discrete
Helmholtz equation in 1, 2, and 3-D. Unit-tested source code and
figure reproduction scripts can be found at
\url{https://www.github.com/ben-bougher/inversions}.

\subsection*{JULIA}
The JULIA programming language is a relatively new paradigm for
scientific computing, and this project served as the author's first
date with JULIA. Built from the ground up for scientific computing,
JULIA makes use of just-in-time compilation and optional static typing
to offer near fortran/C speeds while maintaining a high-level
user-friendly syntax. Julia offers direct interoperability with other
languages and although out of this project scope; a framework for
high-performance parallel computing. Overall JULIA was an intuitive
language to learn; it has reasonable online documentation and it's
easy integration with Python allowed direct use of familiar plotting
libraries.

\subsection*{Operators}
The file operators.jl contains functions for creating gradient, node
averaging, and edge averaging matrices for 1,2, or 3-D data. All
operators assume multi-dimensional data$(mnl)$ is flattened into
Fortran-ordered $N=mnl$ sized vectors.
\subsection*{$Av$ operator}
The $Av$ operator will average each node on a cell to yield an average
value at the cell centre. The 1-D case

\begin{equation}
  \nonumber
  Av_{m,m +1}^1 = \frac{1}{2}
  \begin{pmatrix}
    1 & 1 & 0 & \cdots & \cdots & 0 \\
    0 & 1 & 1 & 0 & \cdots  & 0 \\
    \vdots   && \ddots & \ddots  \\
    0 & 0 & \cdots \cdots & \cdots & 1 & 1
  \end{pmatrix}
\end{equation}
can be generalized to 2-D
\begin{equation}
  \nonumber
  Av_{M, M'}^2 = kron(Av_{n, n+1}^1, Av_{m, m+1}^1)
\end{equation}
and 3-D
\begin{equation}
  \nonumber
  Av_{N,N'}^3 = kron(Av_{l, l+1}^1, Av_{M, M'}^2)
\end{equation}
where
\begin{compactitem}
\item$M=mn$ is the number of 2-D elements
\item$M'=(m+1)(n+1)$ is total number of 2-D nodes
\item$N=mnl$ is the number of 3-D elements
\item$N'$ is the total number of 3D nodes
\item$kron()$ is the Kronecker matrix product
\end{compactitem}
 
\subsection*{$G$ operator}
The gradient operator uses cell nodes to calculate the gradient along
cell edges. The 1-D case
\begin{equation}
  \nonumber
  G_{m,m +1}^1 = \frac{1}{dm}
  \begin{pmatrix}
    -1 & 1 & 0 & \cdots & \cdots & 0 \\
    0 & -1 & 1 & 0 & \cdots  & 0 \\
    \vdots   && \ddots & \ddots  \\
    0 & 0 & \cdots \cdots & \cdots & -1 & 1
  \end{pmatrix}
\end{equation}
can be generalized to 2-D
\begin{equation}
  \nonumber
  G_{M^*, M'}^2 = 
  \begin{pmatrix}
    kron(I_{n+1}, G_{m, m+1}^1) \\
    kron(G_{n,n+1}^1, I_{m+1})
  \end{pmatrix}
\end{equation}
and 3-D
\begin{equation}
  \nonumber
  G_{N^*,N'}^3 = 
  \begin{pmatrix}
    kron(I_{l+1}, kron(I_{n+1}, G_{m,m+1}^1)) \\
    kron(I_{l+1}, kron(G_{n,n+1}^1,I_{m+1})) \\
    kron(G_{n,n+1}, kron(I_{m+1}, I_{n+1})) \\
  \end{pmatrix}
\end{equation}
where
\begin{compactitem}
\item$M^*=m(n+1)+ n(m+1)m$ is the number of 2-D cell edges
\item$N^*=m(n+1)(l+1) + n(m+1)(l+1) + l(m+1)(n+1)$ is the number of
  3-D cell edges
\end{compactitem}

 \subsection*{$Av^e$ operator}
 The $Av^e$ operator sums the averages of the parallel edges of
 cells. The 2-D case
 \begin{equation}
   \nonumber
   Av^{e2}_{M, M^*} = 
   \begin{pmatrix}
     kron(I_{m}, Av_{n,n+1}) \\
     kron(Av_{m, m+1}, I_{n})
   \end{pmatrix}
   ^T
 \end{equation}
 can be extended to the 3-D case
 \begin{equation}
   \nonumber
   Av^{e3}_{N, N^*} = 
   \begin{pmatrix}
     kron(Av_{l,l+1},kron(Av_{n,n+1}, I_{m}))\\
     kron(Av_{n,n+1},kron(I_{n}, Av_{m,m+1}))\\
     kron(kron(I_{l}, Av_{m,m+1}), Av_{n,n+1}))
   \end{pmatrix}
   ^T
 \end{equation}
 which reduces to $I_m$ for the 1-D case.

\subsection*{Unit tests}
The operators were developed using test-driven development, where unit
tests are written before the functions. It encourages the problem to
be divided into testable modules and makes it easier to isolate errors
when working on larger integrated projects. If all of the tests pass,
a higher-level of confidence can be attributed to the code. Tests were
generated by comparing simple manual truth calculations to the output
of each operator. These unit tests were essential for understanding
some of the complicated indexing for higher-dimensional
averaging. Running the file $test.jl$ will run each test and notify
the user of any failures.
% ------------------------------------------------

\section{Method of fictitious sources}
Although each operator is unit-tested to a level of confidence, the
full Helmholtz solution was tested using the method of fictitious
sources \cite{eldad}. The method requires a choice of an analytic
$\hat U$ that satisfies the zero Neumann boundary condition. Applying
the Helmholtz equation to $\hat U$ results in a fictitious source,
$q$. Solving the discrete Helmholtz equation for $U$ given $q$ should
yield $\hat U$ within an error bound. Discretization theory predicts
error that is $O(h^2)$, so doubling the resolution results in a
convergence rate of 4. Figures \ref{fig:1d_comp} - \ref{fig:3d_conv}
show comparisons and convergence rates for 1, 2, and 3-D
implementations. Note that the 2-D case resulted in a convergence rate
of 2, which is problematic as it doesn't agree with predictions from
discretization theory. The 2-D case requires more analysis to
determine to if the convergence is indeed correct.

\begin{figure}[H]
  \centering
  \includegraphics[width=\linewidth]{../figures/figure1.eps}
  \caption{Fictitious source comparison in 1-D using the test field
    $U(x)=cos(\pi x)$.}
  \label{fig:1d_comp}
\end{figure}
\begin{figure}[H]
  \centering
  \includegraphics[width=\linewidth]{../figures/figure2.eps}
  \caption{Rate of convergence for doubling the node doubling using
    the 1-D using the test field $U(x,y)=cos(\pi x)$.}
  \label{fig:1d_conv}
\end{figure}
\begin{figure}[H]
  \centering
  \includegraphics[width=\linewidth]{../figures/figure3.eps}
  \caption{Fictitious source comparison in 2-D using the test field
    $U(x,y)=cos(\pi x)+cos(\pi y)$.}
  \label{fig:2d_comp}
\end{figure}
\begin{figure}[H]
  \centering
  \includegraphics[width=\linewidth]{../figures/figure4.eps}
  \caption{Rate of convergence for node doubling using the 2-D using
    the test field $U(x,y)=cos(\pi x)+cos(\pi y)$.}
  \label{fig:2d_conv}
\end{figure}
\begin{figure}[H]
  \centering
  \includegraphics[width=\linewidth]{../figures/figure5.eps}
  \caption{Fictitious source comparison in 3-D using the test field
    $U(x,y,z)=cos(\pi x)+cos(\pi y)+cos(\pi z)$ at $z=10$.}
  \label{fig:3d_comp}
\end{figure}
\begin{figure}[H]
  \centering
  \includegraphics[width=\linewidth]{../figures/figure6.eps}
  \caption{Rate of convergence for node doubling using the 3-D using
    the test field $U(x,y,z)=cos(\pi x)+cos(\pi y)+cos(\pi z)$.}
  \label{fig:3d_conv}
\end{figure}

\section{Marmousi model}
Since its creation by the Institut Fran�ais du P�trole (IFP) in 1988,
the Marmousi model \cite{marmousi} has become the industry standard
test data for assessing seismic modelling. The model is based on the
North Quenguela trough and its geometry and velocity model yield
complex seismic data.

Qualitative results of the discrete Helmholtz solver applied to the
Marmousi model are shown in Figure \ref{fig:marmousi} and Figure
\ref{fig:marmousi_helmholtz}. A point source was placed at every cell
along the surface to illuminate the model and $w$ was increased to assess
changes in resolution. The resulting wave fields are somewhat difficult to interpret 
but will be used to verify the solver against existing implementations.

\begin{figure}[H]
  \centering
  \includegraphics[width=\linewidth]{../figures/figure7.eps}
  \caption{Marmousi velocity model.}
  \label{fig:marmousi}
\end{figure}

\begin{figure}[H]
  \centering
  \includegraphics[width=\linewidth]{../figures/figure8.eps}
  \caption{Discrete Helmholtz solution for $w=3, w=5, w=10$.}
  \label{fig:marmousi_helmholtz}
\end{figure}

\section{Future work}
The discrete Helmholtz solver is the first step in developing a JULIA
implementation of FWI. A robust Helmholtz solution is a critical
element, but much more work is required. This will include:
\begin{compactitem}
\item an Euler time-stepping implementation of the wave-equation
\item sensitivity tests to assess how the solution changes with
  respect to changes in the model
\item optimizations that can minimize the non-convex objective
  function and invert for the model
\end{compactitem}

% ----------------------------------------------------------------------------------------
% REFERENCE LIST
% ----------------------------------------------------------------------------------------

\begin{thebibliography}{99} % Bibliography - this is intentionally simple in this template

\bibitem{eldad} Eldad Haber, Computational Methods in Geophysical
  Electromagnetics
\bibitem{marmousi} Trevor Irons, Marmousi model,
  \url{http://www.reproducibility.org/RSF/book/data/marmousi/paper_html/paper.html}
\end{thebibliography}

% ----------------------------------------------------------------------------------------

\end{multicols}

\end{document}
